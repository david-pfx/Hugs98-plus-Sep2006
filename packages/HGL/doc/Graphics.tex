%**<title> The Hugs Graphics Library </title>
%**<body bgcolor="#ffffff">
%**<h2>The Hugs Graphics Library (Version 2.0)</h2>
%**<font size=4> <blockquote>
%**Alastair Reid<br>
%**Reid Consulting (UK) Limited<br>
%**alastair@reid-consulting-ltd.ltd.uk<br>
%**</blockquote></font>
%*ignore

%\documentstyle[11pt]{article}
\documentstyle{article}

% copied from the Haskore tutorial
\textheight=8.5in
\textwidth=6.5in
\topmargin=-.3in
\oddsidemargin=0in
\evensidemargin=0in
\parskip=6pt plus2pt minus2pt

\topsep=0pt % how much extra space (on top of parskip) is added round list/verbatim

% and some of my own personal preferences
\parindent=0in

\newcommand{\var}[1]{{\tt #1\/}}    % variables
\newcommand{\fun}[1]{{\tt #1\/}}    % functions
\newcommand{\expr}[1]{{\tt #1\/}}   % expressions
\newcommand{\type}[1]{{\tt #1\/}}   % types
\newcommand{\module}[1]{{\tt #1\/}} % modules
%\newcommand{\modName}[1]{\item[module #1]:}
\newcommand{\modName}[1]{\subsection{Module {#1}}}

\newcommand{\tva}{$\alpha$} % type variables
\newcommand{\tvb}{$\beta $}
\newcommand{\tvc}{$\gamma$}

\newcommand{\arrow}{$\enspace\to\enspace$} % type constructors

\newcommand{\Hugs}{{\sffamily Hugs\/}}
\newcommand{\GHC}{{\sffamily GHC\/}}
\newcommand{\Haskell}{{\sffamily Haskell\/}}
\newcommand{\Library}{{\sffamily Hugs Graphics Library\/}}

\newenvironment{aside}
  {\noindent
   \begingroup
     \small
     {\bf Aside}
     \list{}
          {\topsep 0pt
           \advance\leftmargin -1.5em
           \sl
          }%
     \item\relax%
  }
  {  \endlist
     {\bf End aside.}
   \endgroup
  }

\newenvironment{note}
  {\noindent
   \begingroup
     \small
     {\bf Note}
     \list{}
          {\topsep 0pt
           \advance\leftmargin -1.5em
           \sl
          }%
     \item\relax%
  }
  {  \endlist
     {\bf End note.}
   \endgroup
  }

\newcommand{\Portability}[1]{\par{{\bf Portability Note:} \sl #1}\par}

\newenvironment{portability}{%
%  \medbreak
  \noindent
  \begingroup
    \small
    {\bf Portability Note: }
    \nobreak
    \sl
    \begin{itemize}
    \itemsep0pt
}{%
    \end{itemize}
    \nobreak
    {\bf End Portability Note.}
  \endgroup
%  \medbreak
}

\def\NotInX#1{{#1} is not provided in the X11 implementation of this library.}
\def\NotInWin#1{{#1} is not provided in the Win32 implementation of this library.}

% These are used for reminders, communication between authors, etc.
% There should be no calls to these guys in the final document.

\newcommand{\HeyPaul}[1]{\par{{\bf Hey Paul:} \sl #1}\par}
\newcommand{\ToDo}[1]{\par{{\bf ToDo:} \sl #1}\par}

\newenvironment{outline}{%
%  \medbreak
  \noindent
  {\bf Outline: }
  \begingroup
    \nobreak
    \sl
}{%
  \endgroup
  \nobreak
  {\bf End outline.}
%  \medbreak
}

% Here's how you create figures
%
% \begin{figure*}
% \centerline{
% Foo
% }
% \caption{...}
% \label{...}
% \end{figure*}

\begin{document}

\title{%
  The Hugs Graphics Library\\
  (Version 2.0)%
}

\author{Alastair Reid\\
Reid Consulting (UK) Limited\\
{\tt alastair@reid-consulting-ltd.ltd.uk}\\
{\tt http://www.reid-consulting-uk.ltd.uk/alastair/}}

%\date{26 November, 1999}

\maketitle

%*endignore

\section{Introduction}\label{introduction}

The \Library{} is designed to give the programmer access to
most interesting parts of the Win32 Graphics Device Interface and X11
library without exposing the programmer to the pain and anguish
usually associated with using these interfaces.

To give you a taste of what the library looks like, here is the 
obligatory ``Hello World'' program:

\begin{verbatim}
> module Hello where
>
> import GraphicsUtils
>
> helloWorld :: IO ()
> helloWorld = runGraphics (do
>   w <- openWindow "Hello World Window" (300, 300)
>   drawInWindow w (text (100, 100) "Hello")
>   drawInWindow w (text (100, 200) "World")
>   getKey w
>   closeWindow w
>   )
\end{verbatim}

Here's what each function does:
\begin{itemize}
\item
  \expr{runGraphics :: IO () -> IO ()} get \Hugs{} ready to do graphics,
  runs an action (here, the action is a sequence of 5 subactions) and
  cleans everything up at the end.%
  %
  \footnote{%
    The description of
    \fun{runGraphics} is rather vague because of our promise to protect
    you from the full horror of Win32/X11 programming.  If you really want to
    know, we highly recommend Charles Petzold's book ``Programming
    Windows''~\cite{petzold}
    which does an excellent job with a difficult subject or Adrian Nye's 
    ``Xlib Programming Manual''~\cite{Xlib} which is almost adequate.%
  }
  %

\item
  \expr{openWindow :: Title -> Point -> IO Window} opens a window 
  specifying the window title ``Hello World Window'' and the size
  of the window (300 pixels $\times$ 300 pixels).

\item
  \expr{drawInWindow :: Window -> Graphic -> IO ()} draws a \type{Graphic}
  on a \type{Window}.

\item
  \expr{text :: Point -> String -> Graphic} creates a \type{Graphic}
  consisting of a \type{String} at a given screen location. 

\item
  \expr{getKey :: Window -> IO Char} waits for the user to press
  (and release) a key.  This is necessary to prevent the window
  from closing before you have a chance to read what's on the screen.

\item
  \expr{closeWindow :: Window -> IO ()} closes the window.

\end{itemize}

The rest of this document is organized as follows:
%\begin{itemize}
%\item 
Section~\ref{graphics} describes the \type{Graphic} type (a declarative
    way of drawing pictures);
%\item 
Section~\ref{windows} describes \type{Window}s;
%\item 
Section~\ref{events} describes \type{Event}s;
%\item
Section~\ref{concurrency} describes the Concurrent Haskell
      primitives which you need to create complex interfaces; and
%\item 
Section~\ref{Draw} describes the \type{Draw} monad (a more
      imperative side to the \type{Graphic} type).
%\end{itemize}

% \begin{note}
% This document is just a draft --- it contains British spelling (which
% conflicts horribly with the US spelling used within the library); it
% is incomplete; and it probably contains the occasional error as well.
% \end{note}

%%%%%%%%%%%%%%%%%%%%%%%%%%%%%%%%%%%%%%%%%%%%%%%%%%%%%%%%%%%%%%%%

\section{Graphics}\label{graphics}

In section~\ref{introduction}, we used these two functions to draw 
to a window

\begin{verbatim}
> drawInWindow :: Window -> Graphic -> IO ()
> text         :: Point  -> String  -> Graphic
\end{verbatim}

This section describes other ways of creating graphics that can be
drawn to the screen.

\subsection{Atomic Graphics}\label{primitives}

Here's a list of the atomic operations

\begin{verbatim}
> emptyGraphic ::                                     Graphic
> ellipse      :: Point -> Point                   -> Graphic
> shearEllipse :: Point -> Point -> Point          -> Graphic
> arc          :: Point -> Point -> Angle -> Angle -> Graphic
> line         :: Point -> Point                   -> Graphic
> polyline     :: [Point]                          -> Graphic
> polygon      :: [Point]                          -> Graphic
> polyBezier   :: [Point]                          -> Graphic
> text         :: Point -> String                  -> Graphic
\end{verbatim}

\fun{emptyGraphic} is a blank \type{Graphic}.

\fun{ellipse} is a filled ellipse which fits inside a rectangle
defined by two \type{Point}s on the window. \fun{shearEllipse} is a
filled ellipse inside a parallelogram defined by three \type{Point}s
on the window.  \fun{arc} is an unfilled elliptical arc which fits
inside a rectangle defined by two \type{Point}s on the window.  The
angles specify the start and end points of the arc --- the arc
consists of all points from the start angle counter-clockwise to the
end angle.  Angles are in degrees $[0..360]$ rather than radians
$[0..2\pi]$.

\fun{line} is a line between two \type{Point}s.  \fun{polyline}
is a series of lines through a list of \type{Point}s.
\fun{polyBezier} is a series of (unfilled) bezier curves defined by
a list of $3n+1$ control \type{Point}s.
\fun{polygon} is a filled polygon defined by a list of \type{Point}s.

\fun{text} is a rendered \type{String}.

\begin{portability}
%\begin{itemize}
\item \NotInX{\fun{polyBezier}}
\item \fun{shearEllipse} is implemented by polygons on both Win32 and X11.
%\end{itemize}
\end{portability}


\subsection{Graphic Modifiers}\label{modifiers}

One of the most useful properties of \type{Graphic}s is that they can
be modified in various ways.  Here is a selection of the modifiers
available

\begin{verbatim}
> withFont          :: Font      -> Graphic -> Graphic
> withTextColor     :: RGB       -> Graphic -> Graphic
> withTextAlignment :: Alignment -> Graphic -> Graphic
> withBkColor       :: RGB       -> Graphic -> Graphic
> withBkMode        :: BkMode    -> Graphic -> Graphic
> withPen           :: Pen       -> Graphic -> Graphic
> withBrush         :: Brush     -> Graphic -> Graphic
> withRGB           :: RGB       -> Graphic -> Graphic
\end{verbatim}

The effect of these ``modifiers'' is to modify the way in which a
graphic will be drawn.  For example, if \var{courier :: Font} is a 10
point Courier font, then drawing \expr{withFont courier (text (100,100)
"Hello")} will draw the string \expr{"Hello"} on the window using the
10 point Courier font.

Modifiers are cumulative: a series of modifiers can be applied to a
single graphic.  For example, the graphic
  
\begin{verbatim}
> withFont courier (
>   withTextColor red (
>     withTextAlignment (Center, Top) (
>       text (100,100) "Hello World"
>     )
>   )
> )
\end{verbatim}

will be 
%
\begin{itemize}
\itemsep0pt
\item horizontally aligned
  so that the centre of the text is at \expr{(100, 100)};

\item vertically aligned
  so that the top of the text is at \expr{(100, 100)};

\item colored red

\item displayed in 10 point Courier font

\end{itemize}

Modifiers nest in the obvious way --- so 
%
\begin{verbatim}
> withTextColor red (
>   withTextColor green (
>     text (100,100) "What Color Am I?"
>   )
> )
\end{verbatim}
%
will produce green text, as expected.

\begin{aside}
As you write more and more complex graphics, you'll quickly realize
that it's very tedious to insert all those parentheses and to keep
everything indented in a way that reveals its structure.

Fortunately, the Haskell Prelude provides a right associative 
application operator
%
\begin{verbatim}
> ($) :: (a -> b) -> a -> b
\end{verbatim}
%
which eliminates the need for almost all parentheses when defining
\type{Graphic}s.  Using the \fun{(\$)} operator, the above example can
be rewritten like this
%
\begin{verbatim}
> withTextColor red   $
> withTextColor green $
> text (100,100) "What Color Am I?"
\end{verbatim}
%

\end{aside}

\subsection{Combining Graphics}\label{combining}

The other useful property of \type{Graphic}s is that they can be
combined using the \fun{overGraphic} combinator

\begin{verbatim}
> overGraphic :: Graphic -> Graphic -> Graphic
\end{verbatim}

For example, drawing this graphic produces a red triangle ``on top of''
(or ``in front of'') a blue square
%
\begin{verbatim}
> overGraphic
>   (withBrush red  $ polygon [(200,200),(400,200),(300,400)])
>   (withBrush blue $ polygon [(100,100),(500,100),(500,500),(100,500)])
\end{verbatim}

Notice that modifiers respect the structure of a graphic --- modifiers
applied to one part of a graphic have no effect on other parts of
the graphic.  For example the above graphic could be rewritten like
this.
%
\begin{verbatim}
> withBrush blue $ 
> overGraphic
>   (withBrush red  $ polygon [(200,200),(400,200),(300,400)])
>   (polygon [(100,100),(500,100),(500,500),(100,500)])
\end{verbatim}

The \fun{overGraphics} function is useful if you want to draw a list of
graphics.  It's type and definition are

\begin{verbatim}
> overGraphics :: [Graphic] -> Graphic
> overGraphics = foldr overGraphic emptyGraphic
\end{verbatim}

Notice that graphics at the head of the list are drawn ``in front of'' 
graphics at the tail of the list.


\subsection{Attribute Generators}\label{generators}

The graphic modifiers listed at the start of Section~\ref{modifiers}
use attributes with types like \type{Font}, \type{RGB} and
\type{Brush}, but so far we have no way of generating any of these attributes.

Some of these types are {\em concrete\/} (you can create them using
normal data constructors) and some are {\em abstract\/} (you can only create
them with special ``attribute generators'').  Here's the definitions
of the concrete types.

\begin{verbatim}
> type Angle     = Double
> type Dimension = Int
> type Point     = (Dimension,Dimension)
> data RGB       = RGB Int Int Int
> 
> -- Text alignments
> type Alignment = (HAlign, VAlign)
> -- names have a tick to distinguish them from Prelude names (blech!)
> data HAlign = Left' | Center   | Right'
> data VAlign = Top   | Baseline | Bottom
> 
> -- Text background modes
> data BkMode = Opaque | Transparent
\end{verbatim}

The attributes \type{Font}, \type{Brush} and \type{Pen} are
{\em abstract,\/} and are a little more complex because we want to 
delete the font, brush, or pen once we've finished using it.  This
gives the attribute generators a similar flavour to the modifiers seen
in section~\ref{modifiers} --- these functions are applied to an
argument of type \type{\tva \arrow Graphic} and return a
\type{Graphic}.

%
\begin{verbatim}
> mkFont  :: Point -> Angle -> Bool -> Bool -> String -> 
>                                   (Font  -> Graphic) -> Graphic
> mkBrush ::                 RGB -> (Brush -> Graphic) -> Graphic
> mkPen   :: Style -> Int -> RGB -> (Pen   -> Graphic) -> Graphic
\end{verbatim}

For example, the following program uses a $50 \times 50$ pixel, non-bold,
italic, courier font to draw red text on a green background at an angle
of 45 degrees across the screen.

\begin{verbatim}
> fontDemo = runGraphics $ do
>   w <- openWindow "Font Demo Window" (100,100)
>   drawInWindow w $
>     withTextColor (RGB 255 0 0)                 $
>     mkFont (50,100) (pi/4) False True "courier" $ \ font ->
>     withFont font                               $
>     withBkColor (RGB 0 255 0)                   $
>     withBkMode  Opaque                          $
>     text (50,50) "Font Demo"
>   getKey w
>   closeWindow w
\end{verbatim}

A default font is substituted if the requested font does not exist.
The rotation angle is ignored if the font is not a ``TrueType'' font
(e.g., for {\tt System} font on Win32).

\begin{portability}
%\begin{itemize}
\item X11 does not directly support font rotation so \fun{mkFont}
always ignores the rotation angle argument in the X11 implementation
of this library.

\item Many of the font families typically available on Win32 are not
available on X11 (and {\it vice-versa\/}).  In our experience, the
font families ``courier,'' ``helvetica'' and ``times'' are somewhat
portable.
%\ToDo{Check this}

%\end{itemize}
\end{portability}



\subsection{Brushes, Pens and Text Colors}

If you were counting, you'll have noticed that there are five separate
ways of specifying colors

\begin{verbatim}
> mkBrush       ::                 RGB -> (Brush -> Graphic) -> Graphic
> mkPen         :: Style -> Int -> RGB -> (Pen   -> Graphic) -> Graphic
> withTextColor ::                 RGB -> Graphic            -> Graphic
> withBkColor   ::                 RGB -> Graphic            -> Graphic
> withRGB       ::                 RGB -> Graphic            -> Graphic
\end{verbatim}

What do these different modifiers and attributes control?

\begin{description}
\item[Brushes] 
  are used when filling shapes --- so the brush color is
  used when drawing polygons, ellipses and regions.

\item[Pens]
  are used when drawing lines --- so the pen color is used
  when drawing arcs, lines, polylines and polyBeziers.

  Pens also have a ``style'' and a ``width''.  The \type{Style}
  argument is used to select solid lines or various styles of
  dotted and dashed lines.

\begin{verbatim}
> data Style
>   = Solid 
>   | Dash        -- "-------"
>   | Dot         -- "......."  
>   | DashDot     -- "_._._._"  
>   | DashDotDot  -- "_.._.._"  
>   | Null
>   | InsideFrame
\end{verbatim}

\item[TextColor]
  is used as the foreground color when drawing text.

\item[BkColor]
  is used as the background color when drawing text with
  background mode \fun{Opaque}.  The background color is
  ignored when the mode is \fun{Transparent}.
  
%  \ToDo{Should I expand the name \type{BkColor} to \type{BackgroundColor}?}

\end{description}

Finally, \fun{withRGB} is a convenience function which sets the brush,
pen and text colors to the same value.  Here is its definition

\begin{verbatim}
> withRGB :: RGB -> Graphic -> Graphic
> withRGB c g = 
>   mkBrush c       $ \ brush ->
>   withBrush brush $
>   mkPen Solid 2 c $ \ pen ->
>   withPen pen     $
>   withTextColor c $
>   g
\end{verbatim}

\begin{portability}
%\begin{itemize}
\item
  On Win32, the pen is also used to draw a line round all
  the filled shapes --- so the pen color also affects how
  polygons, ellipses and regions are drawn.

\item
  One of the Win32 ``gotchas'' is that the choice of \type{Style}
  only applies if the width is 1 or less.  With greater widths,
  the pen style will always be \fun{Solid} no matter what you try to
  select.  This problem does not apply to X11.
%\end{itemize}
\end{portability}


\subsection{Named Colors}

Working with RGB triples is a pain in the neck so the
\module{HGL.Graphics.Utils} module provides these built in colors as
convenient ``abbreviations.''

\begin{verbatim}
> data Color 
>   = Black
>   | Blue
>   | Green 
>   | Cyan
>   | Red 
>   | Magenta
>   | Yellow
>   | White
>  deriving (Eq, Ord, Bounded, Enum, Ix, Show, Read)
\end{verbatim}

This type is useful because it may be used to index an array of
RGB triples.

\begin{verbatim}
> colorTable :: Array Color RGB
\end{verbatim}

For example, we provide this function which looks up a color in the
\var{colorTable} and uses that color for the brush, pen and text color.

\begin{verbatim}
> withColor :: Color -> Graphic -> Graphic
\end{verbatim}

It's worth pointing out that there's nothing ``magical'' about the
\type{Color} type or our choice of colors.  If you don't like our
choice of colors, our names, or the way we mapped them onto RGB
triples, you can write your own!  To get you started, here's our
implementation of \fun{withColor} and \var{colorTable}.

\begin{verbatim}
> withColor c = withRGB (colorTable ! c)
> 
> colorTable = array (minBound, maxBound) colorList
> 
> colorList  :: [(Color, RGB)]
> colorList =
>   [ (Black   , RGB   0   0   0)
>   , (Blue    , RGB   0   0 255)
>   , (Green   , RGB   0 255   0)
>   , (Cyan    , RGB   0 255 255)
>   , (Red     , RGB 255   0   0)
>   , (Magenta , RGB 255   0 255)
>   , (Yellow  , RGB 255 255   0)
>   , (White   , RGB 255 255 255)
>   ]
\end{verbatim}

\subsection{Bitmaps}

\type{Bitmap}s can be displayed in three ways:

\begin{enumerate}
\item with no transformation at  a point
\item stretched to fit           a rectangle
\item rotated and sheared to fit a parallelogram
\end{enumerate}

Rectangles are specified by a pair of points: the top-left, and
bottom-right corners of the rectangle.

\begin{verbatim}
> bitmap        :: Point                      -> Bitmap -> Graphic
> stretchBitmap :: Point  -> Point            -> Bitmap -> Graphic
> shearBitmap   :: Point  -> Point  -> Point  -> Bitmap -> Graphic
\end{verbatim}

\type{Bitmap}s are read in from files and disposed of using 
\begin{verbatim}
> readBitmap    :: String -> IO Bitmap
> deleteBitmap  :: Bitmap -> IO ()
\end{verbatim}
(but be sure that the current \type{Graphic} on a \type{Window}
doesn't contain a reference to a \type{Bitmap} before you delete the
\type{Bitmap}!)

This operation gets the size of a bitmap.
\begin{verbatim}
> getBitmapSize :: Bitmap -> IO (Int, Int)
\end{verbatim}

% \ToDo{%
%   Describe the other bitmap operations 
%   --- clean them up a bit first though!%
% }

\begin{portability}
%\begin{itemize}
\item The Bitmap functions are not currently provided in the X11
implementation of this library.
\item \fun{shearBitmap} is supported on Win'NT but not Win'95.
%\end{itemize}
\end{portability}



\subsection{Regions}

\type{Region}s can be viewed as an efficient representation of sets of
pixels.  They are created from rectangles, ellipses, polygons and
combined using set operations (intersection, union, difference and xor
(symmetric difference)).

These are the operations available:
\begin{verbatim}
> emptyRegion     ::                   Region
> rectangleRegion :: Point -> Point -> Region
> ellipseRegion   :: Point -> Point -> Region
> polygonRegion   :: [Point]        -> Region
> 
> intersectRegion :: Region -> Region -> Region
> unionRegion     :: Region -> Region -> Region
> subtractRegion  :: Region -> Region -> Region
> xorRegion       :: Region -> Region -> Region
>                            
> regionToGraphic :: Region -> Graphic
\end{verbatim}

\fun{withBrush} affects the color of \fun{regionToGraphic}.

\begin{portability}
%\begin{itemize}
\item \NotInWin{\fun{emptyRegion}}
It is possible to use an empty rectangle region instead
\item \fun{ellipseRegion} is implemented using polygons in the X11
implementation of the library.
%\end{itemize}
\end{portability}


\subsection{The \type{Graphic} Algebra}

The Graphic modifiers satisfy a large number of useful identities.
For example, 

\begin{itemize}
\item
  The triple $\langle \type{Graphic}, \fun{overGraphic}, \fun{emptyGraphic} \rangle$
  forms a ``monoid.'' If this wasn't true, we wouldn't find the
  \fun{overGraphics} function very useful.

\item
  Modifiers and generators all distribute over \fun{overGraphic}.
  That is,
\begin{verbatim}
> mkFoo <args> (p1 `overGraphic` p2) 
>  = (mkFoo <args> p1) `overGraphic` (mkFoo <args> p2)
> withFoo foo  (p1 `overGraphic` p2) 
>  = (withFoo foo  p1) `overGraphic` (withFoo foo  p2)
\end{verbatim}

(These laws are especially useful when trying to make programs more efficient
--- see section~\ref{efficiency}.)

\item
  ``Independent'' modifiers commute with each other.
  For example,
\begin{verbatim}
> withTextColor c (withTextAlignment a p) 
>  = withTextAlignment a (withTextColor c p)
\end{verbatim}

\item
  Generators commute with modifiers.
  For example,
\begin{verbatim}
> mkBrush c (\ b -> withBrush b' p) = withBrush b' mkBrush c (\ b -> p)
\end{verbatim}
  if \var{b} and \var{b'} are distinct.

\item
  Generators commute with other generators.  For example

\begin{verbatim}
> mkBrush c (\ b -> mkBrush c' (\ b' -> p)) 
>  = mkBrush c' (\ b' -> mkBrush c (\ b -> p))
\end{verbatim}
  if \var{b} and \var{b'} are distinct.

\item
  ``Irrelevant'' modifiers and generators can be added or removed at will.
  For example, the text color has no effect on line drawing
\begin{verbatim}
> withTextColor c (line p0 p1) = line p0 p1
\end{verbatim}
  and there's no need to create a brush if you don't use it
\begin{verbatim}
> mkBrush c (\ b -> p) = p, if b does not occur in p
\end{verbatim}

This last law can also be stated in the form
\begin{verbatim}
> mkBrush c (\ b -> atomic) = atomic
\end{verbatim}
for any atomic operation.

\end{itemize}

% \ToDo{%
%   Add tables describing which modifiers are relevant and which
%   ones are independent.  Or just list all the identities.%
% }

The practical upshot of all this is that there are many ways to
rearrange a graphic so that it will be drawn more (or less)
efficiently.  We explore this topic in the next section.


\subsection{Efficiency Considerations}\label{efficiency}

The other sections provide a very simple set of functions for creating
graphics --- but at the cost of ignoring efficiency.  For example, this
innocent looking graphic

\begin{verbatim}
> overGraphics
>   [ withColor Red $ ellipse (000,000) (100,100)
>   , withColor Red $ ellipse (100,100) (200,200)
>   , withColor Red $ ellipse (200,200) (300,300)
>   ]
\end{verbatim}

will take longer to draw than this equivalent graphic

\begin{verbatim}
> mkBrush (colorTable ! Red) $ \ redBrush  ->
> overGraphics
>   [ withBrush redBrush $ ellipse (000,000) (100,100)
>   , withBrush redBrush $ ellipse (100,100) (200,200)
>   , withBrush redBrush $ ellipse (200,200) (300,300)
>   ]
\end{verbatim}

Briefly, the problems are that \fun{withColor} sets the color of the
brush, the pen and the text but ellipses only use the brush color;
and we're calling \fun{withColor} $3$ times more than we have to.
This wouldn't matter if brush creation was cheap and easy.  However,
most typical workstations can only display at most $256$ or $65536$
different colors on the screen at once but allow you to specify any
one of $16777216$ different colors when selecting a drawing color
--- finding a close match to the requested color can be as
expensive as drawing the primitive object itself.

% \ToDo{%
%   Time both graphics --- check that my estimate is about right.%
% }

This doesn't matter much for a graphic of this size --- but if you're
drawing several thousand graphic elements onto the screen as part of
an animation, it can make the difference between a quite respectable
frame rate of 20--30 frames per second and an absolutely unusable
frame rate of 2--3 frames per second.

% \begin{aside}
% 
% The (lazy) functional programming community has a bad habit of
% ignoring these kinds of considerations; with the result that C
% programmers have acquired the notion that functional programs will
% {\em necessarily\/} run {\em several orders of magnitude\/} more
% slowly than equivalent C programs.  On the basis of this notion, 
% they {\em quite rightly\/} regard functional languages as toys which
% have no relevance to Real Programming.
% 
% We'd like to dispel that belief and so, in designing the graphics
% library, we have made a serious attempt to expose enough of the
% underlying machinery that we can tackle this sort of efficiency
% consideration.
% 
% That said, it's worth emphasising that \Hugs{} is an interpreter which
% makes it run 10--100 times more slowly than compiled implementations
% of \Haskell{} such as \GHC{}.  If you really are wanting to animate
% thousands of objects, you probably shouldn't be relying on \Hugs{}.
% 
% \end{aside}


\subsubsection{Eliminate calls to \fun{withRGB} and \fun{withColor}}

At the risk of pointing out the obvious, the first step in optimizing
a program in this way is to expand all uses of the \fun{withRGB} and
\fun{withColor} functions and eliminating unnecessary calls to
\fun{mkBrush}, \fun{mkPen} and \fun{withTextColor}.  Applying this
optimization to the above \type{Graphic}, we obtain this (which should run
about 3 times faster).

\begin{verbatim}
> overGraphics
>   [ mkBrush red $ \ redBrush -> withBrush redBrush $ ellipse (00,00) (10,10)
>   , mkBrush red $ \ redBrush -> withBrush redBrush $ ellipse (10,10) (20,20)
>   , mkBrush red $ \ redBrush -> withBrush redBrush $ ellipse (20,20) (30,30)
>   ]
\end{verbatim}

\subsubsection{Lifting generators to the top of \type{Graphics}}


Another important optimization is to avoid creating many identical
brushes, pens or fonts when one will do.  We do this by ``lifting''
brush creation out to the top of a graphic.  For example, this graphic

\begin{verbatim}
> overGraphics
>   [ mkBrush red $ \ redBrush -> withBrush redBrush $ ellipse (00,00) (10,10)
>   , mkBrush red $ \ redBrush -> withBrush redBrush $ ellipse (10,10) (20,20)
>   , mkBrush red $ \ redBrush -> withBrush redBrush $ ellipse (20,20) (30,30)
>   ]
\end{verbatim}

creates three red brushes.  It would be more efficient to rewrite it 
like this

\begin{verbatim}
> mkBrush red  $ \ redBrush  ->
> overGraphics
>   [ withBrush redBrush $ ellipse (00,00) (10,10)
>   , withBrush redBrush $ ellipse (10,10) (20,20)
>   , withBrush redBrush $ ellipse (20,20) (30,30)
>   ]
\end{verbatim}

If your program uses a lot of brushes, it may be more convenient to
store the brushes in a ``palette'' (i.e., an array of brushes)

\begin{verbatim}
> mkBrush red  $ \ redBrush  ->
> mkBrush blue $ \ blueBrush ->
> let palette = array (minBound, maxBound) 
>                     [(Red, redBrush), (Blue, blueBrush)]
> in
> overGraphics
>   [ withBrush (palette ! Red)  $ ellipse (00,00) (10,10)
>   , withBrush (palette ! Blue) $ ellipse (10,10) (20,20)
>   , withBrush (palette ! Red)  $ ellipse (20,20) (30,30)
>   ]
\end{verbatim}

% \ToDo{%
%   Write the obvious function with type
%   \type{[RGB] -> ([Brush] -> Graphic) -> Graphic} 
%   (and similarily, for Pens, Fonts, etc).%
% }

\subsubsection{Lifting generators out of graphics}

% \ToDo{Update this section}

Even this program has room for improvement: every time the graphic is
redrawn (e.g., whenever the window is resized), it will create fresh
brushes with which to draw the graphic.  The graphics library provides
a way round this --- but it's more difficult and fraught with danger.

\begin{outline}
This section will talk about using explicit creation and deletion
functions to create brushes, fonts, etc.

The situation isn't very happy at the moment because it's easy to
forget to deallocate brushes before you quit or to deallocate them
before you change the graphic.  

% \ToDo{Maybe things will be better by the time we do an official release...}

\end{outline}



%%%%%%%%%%%%%%%%%%%%%%%%%%%%%%%%%%%%%%%%%%%%%%%%%%%%%%%%%%%%%%%%

\section{Windows}\label{windows}

In section~\ref{introduction} we saw the function \fun{drawInWindow} for
drawing a \type{Graphic} on a \type{Window}.  It turns out that
\fun{drawInWindow} is not a primitive function but, rather, it is defined
using these two primitive functions which read the current
\type{Graphic} and set a new \type{Graphic}.

\begin{verbatim}
> getGraphic   :: Window -> IO Graphic
> setGraphic   :: Window -> Graphic -> IO ()
\end{verbatim}

Here's how these functions are used to define the function \fun{drawInWindow}
(which we used in section~\ref{introduction}) and another useful
function \fun{clearWindow}.

\begin{verbatim}
> drawInWindow :: Window -> Graphic -> IO ()
> drawInWindow w p = do
>   oldGraphic <- getGraphic w
>   setGraphic w (p `over` oldGraphic)
>
> clearWindow :: Window -> IO ()
> clearWindow w = setGraphic w emptyGraphic
\end{verbatim}

% \ToDo{This is no longer true.  I think I want to make the ability to draw
% the delta explicit.}

%%%%%%%%%%%%%%%%%%%%%%%%%%%%%%%%%%%%%%%%%%%%%%%%%%%%%%%%%%%%%%%%

\section{Events}\label{events}

The graphics library supports several different input devices (the
mouse, the keyboard, etc) each of which can generate several different
kinds of event (mouse movement, mouse button clicks, key presses, key
releases, window resizing, etc.)


\subsection{Keyboard events}

In section~\ref{introduction} we saw the function \fun{getKey} being
used to wait until a key was pressed and released.  The function
\fun{getKey} is defined in terms of a more general function \fun{getKeyEx}

\begin{verbatim}
> getKeyEx     :: Window -> Bool -> IO Char
\end{verbatim}

which can be used to wait until a key is pressed (\expr{getKeyEx w True})
or until it is released (\expr{getKeyEx w False}).  The definition of
\fun{getKey} using this function is trivial:

\begin{verbatim}
> getKey       :: Window -> IO Char
> getKey w = do{ getKeyEx w True; getKeyEx w False }
\end{verbatim}


\subsection{Mouse events}

As well as waiting for keyboard events, we can wait for mouse button
events.  We provide three functions for getting these events.
\fun{getLBP} and \fun{getRBP} are used to wait for left and right
button presses.  Both functions are defined using \fun{getButton}
which can be used to wait for either the left button or the right
button being either pressed or released.

\begin{verbatim}
> getLBP       :: Window -> IO Point
> getRBP       :: Window -> IO Point
> getButton    :: Window -> Bool -> Bool -> IO Point
>
> getLBP w = getButton w True  True
> getRBP w = getButton w False True
\end{verbatim}


\subsection{General events}

The functions \fun{getKeyEx} and \fun{getButton} described in the
previous sections are not primitive functions.  Rather they are
defined using the primitive function \fun{getWindowEvent}

\begin{verbatim}
> getWindowEvent :: Window -> IO Event
\end{verbatim}

which waits for the next ``event'' on a given \type{Window}.
\type{Event}s are defined by the following data type.

\begin{verbatim}
> data Event 
>   = Key       { char :: Char, isDown :: Bool }
>   | Button    { pt :: Point, isLeft, isDown :: Bool }
>   | MouseMove { pt :: Point }
>   | Resize
>   | Closed
>  deriving Show 
\end{verbatim}

These events are:

\begin{itemize}
\item
  \expr{Key\{char, isDown\}} occurs when a key is pressed 
  (\expr{isDown==True})
  or released (\expr{isDown==False}).  \expr{char} is the ``keycode'' for the
  corresponding key.  This keycode can be a letter, a number or some other 
  value corresponding to the shift key, control key, etc.  

%   \ToDo{%
%     Say more about what the keycode is --- in the meantime, users will
%     just have to try a few experiments to find out which code each key
%     produces.%
%   }

\item
  \expr{Button\{pt, isLeft, isDown\}} occurs when a mouse button is
  pressed (\expr{isDown==True}) or released (\expr{isDown==False}).
  \expr{pt} is the mouse position when the button was pressed and
  \expr{isLeft} indicates whether it was the left or the right button.

\item
  \expr{MouseMove\{pt\}} occurs when the mouse is moved inside the window.
  \expr{pt} is the position of the mouse after the movement.

\item
  \expr{Resize} occurs when the window is resized.
  The new window size can be discovered using these functions.

\begin{verbatim}
> getWindowRect   :: Window -> IO (Point, Size)
> getWindowSize   :: Window -> IO Size
> getWindowSize w = do
>   (pt,sz) <- getWindowRect w
>   return sz
\end{verbatim}

\item
  \expr{Resize} occurs when the window is closed.

\end{itemize}

\begin{portability}
%\begin{itemize}
\item
  Programmers should assume that the \type{Event} datatype will be extended in the
  not-too-distant future and that individual events may change slightly.
  As a minimum, you should add a ``match anything'' alternative to
  any function which pattern matches against \type{Event}s.

\item
X11 systems typically have three button mice.  Button 1 is used as the
left button, button 3 as the right button and button 2 (the middle
button) is ignored.

%\end{itemize}
\end{portability}

As examples of how \fun{getWindowEvent} might be used in a program, here are
the definitions of \fun{getKeyEx} and \fun{getButton}.

\begin{verbatim}
> getKeyEx     :: Window -> Bool -> IO Char
> getKeyEx w down = loop
>  where
>   loop = do
>         e <- getWindowEvent w
>         case e of 
>           Key{ char = c, isDown } 
>             |  isDown == down 
>             -> return c
>           _ -> loop
\end{verbatim}

\begin{verbatim}
> getButton    :: Window -> Bool -> Bool -> IO Point
> getButton w left down = loop
>  where
>   loop = do
>         e <- getWindowEvent w
>         case e of 
>           Button{pt,isLeft,isDown} 
>             | isLeft == left && isDown == down
>             -> return pt
>           _ -> loop
\end{verbatim}

\subsection{Using Timers}

% \ToDo{%
%   Timers are not very well integrated with the rest of the library at
%   the moment.  We plan to improve this situation in future versions.%
% }

If you want to use a timer, you have to open the window using
\fun{openWindowEx} instead of \fun{openWindow}

\begin{verbatim}
> openWindowEx :: Title -> Maybe Point -> Size ->
>                 RedrawMode -> Maybe Time -> IO Window
>
> data RedrawMode
>   = Unbuffered
>   | DoubleBuffered
\end{verbatim}

This {\em extended\/} version of \fun{openWindow} takes extra parameters
which specify
\begin{itemize}
\item the initial position of a window;
\item how to display a graphic on a window; and
\item the time between ticks (in milliseconds).
\end{itemize}

The function \fun{openWindow} is defined using \fun{openWindowEx}
\begin{verbatim}
> openWindow name size = openWindowEx name Nothing size Unbuffered Nothing
\end{verbatim}

The drawing mode can be either \fun{DoubleBuffered} which uses a
``double buffer'' to reduce flicker or \fun{Unbuffered} which draws
directly to the window and runs slightly faster but is more prone to
flicker.  You should probably use \fun{DoubleBuffered} for
animations.

The timer generates ``tick events'' at regular intervals.  The
function \fun{getWindowTick} waits for the next ``tick event'' to occur.

\begin{verbatim}
> getWindowTick :: Window -> IO ()
\end{verbatim}

\begin{aside}
With normal events, like button presses, we store every event that
happens until you remove that event from the queue.  If we did this
with tick events, and your program takes a little too long to draw
each frame of an animation, the event queue could become so swamped
with ``ticks'' that you'd never respond to user input.  To avoid this
problem, we only insert a tick into the queue if there's no tick there 
already.
\end{aside}

Here's a simple example of how to use timers.  Note the use of
\fun{setGraphic} instead of \fun{drawInWindow}.

\begin{verbatim}
> timerDemo = do
>   w <- openWindowEx 
>          "Timer demo"         -- title
>          (Just (500,500))     -- initial position of window
>          (100,100)            -- initial size of window
>          DoubleBuffered       -- drawing mode - see above
>          (Just 50)            -- tick rate
>   let
>     loop x = do
>       setGraphic w $ text (0,50) $ show x
>       getWindowTick w         -- wait for next tick on window
>       loop (x+1)
>   loop 0
\end{verbatim}

% \ToDo{There is currently no way to specify a background color.}

%%%%%%%%%%%%%%%%%%%%%%%%%%%%%%%%%%%%%%%%%%%%%%%%%%%%%%%%%%%%%%%%

\section{Concurrent Haskell}\label{concurrency}

If you want to use multiple windows or each window contains a
number of essentially independent components, it is convenient
to use separate threads for handling each window.  \Hugs{}
provides a simple mechanism for doing that. 

The simplest concurrency primitives are \fun{par} and \fun{par\_}

\begin{verbatim}
> par  :: IO a -> IO b -> IO (a,b)
> par_ :: IO a -> IO b -> IO ()
\end{verbatim}

(These are both exported from the \module{HGL.Graphics.Utils} module.)

These run two \type{IO} actions in parallel and terminate when both
actions terminate.  The function \fun{par\_} discards the results of
the actions.

\begin{aside}
The underscore in the name \fun{par\_} is derived from the use of
the underscore in the definition of \fun{par\_}.
 
\begin{verbatim}
> par_ p q = (p `par` q) >>= \ _ -> return ()
\end{verbatim}

This naming convention is also used in the Haskell Prelude and
standard libraries (\fun{mapM\_}, \fun{zipWithM\_}, etc.).
\end{aside}

The function \fun{parMany} generalizes \fun{par\_} to lists.

\begin{verbatim}
> parMany :: [ IO () ] -> IO ()
> parMany = foldr par_ (return ())
\end{verbatim}

Of course, you'll quickly realise that there's not much point in being
able to create concurrent threads if threads can't communicate with
each other.  \Hugs{} provides an implementation of the ``Concurrent
Haskell'' primitives described in the Concurrent Haskell
paper~\cite{concurrentHaskell:popl96} to which we refer the enthusiastic reader.

% \begin{aside}
% Programmers should be aware that there is one significant difference
% between \Hugs{}' implementation of concurrency and \GHC{}'s.
% 
% \begin{description}
% \item[GHC]
%   uses preemptive multitasking.
% 
%   Context switches can occur at any time (except if you call a C
%   function (like "getchar") which blocks the entire process while
%   waiting for input.
% 
% \item[Hugs]
%   uses cooperative multitasking.  
% 
%   Context switches only occur when you use one of the primitives
%   defined in this module.  This means that programs such as:
% 
% \begin{verbatim}
% > main = forkIO (write 'a') >> write 'b'
% >  where
% >   write c = putChar c >> write c
% \end{verbatim}
% 
%   will print either "aaaaaaaaaaaaaa..." or "bbbbbbbbbbbb..."
%   instead of some random interleaving of 'a's and 'b's.
% 
% \end{description}
% 
% Cooperative multitasking is sufficient for writing coroutines and simple
% graphical user interfaces. 
% \end{aside}

%%%%%%%%%%%%%%%%%%%%%%%%%%%%%%%%%%%%%%%%%%%%%%%%%%%%%%%%%%%%%%%%

\section{The \type{Draw} monad}\label{Draw}

The \type{Graphic} type, operations and combinators provide a
flexible, efficient and convenient way of drawing images on a window
and encapsulate good programming practice by cleaning up any changes
they must make to the state of the window.  In some applications
though, it is appropriate to use a lower-level, more error-prone
interface for drawing images.  For example, when building a library on
top of the Graphics library, one might want to build on a slightly
more efficient, less secure interface.  Or, when teaching courses on
computer graphics, it would not be possible to demonstrate low-level
aspects of graphics using an interface which hides those aspects.
This section describes the \type{Draw} monad (an imperative graphics
interface) and describes how this is used to implement the
\type{Graphic} type (a declarative graphics interface).  This section
can be ignored by most readers.

\subsection{The \type{Draw} monad and the \type{Graphic} type}\label{Draw monad}

The \type{Graphic} type lets you describe what an image should look
like; the \type{Draw} monad lets you describe how to build an image.
These views intersect for atomic graphics.  For example, the function
to draw a line can serve both as a description and as the
implementation.  This is exploited in the graphics library by defining
\type{Graphic} as an instance of the \type{Draw} monad.  Thus, all
\type{Graphic} types and operations listed in section~\ref{graphics}
can also be used with the \type{Draw} monad.

\begin{verbatim}
> data Draw a = ...
> instance Functor Draw where ...
> instance Monad   Draw where ...
>
> type Graphic = Draw ()
\end{verbatim}

The \fun{emptyGraphic} and \fun{overGraphic} functions are implemented
using this monad.  Their definitions should not be surprising.
\begin{verbatim}
> emptyGraphic        = return ()
> g1 `overGraphic` g2 = g2 >> g1
\end{verbatim}

\subsection{\type{Draw} modifiers and generators}\label{Draw modifiers}

The difference between the \type{Draw} monad and the \type{Graphic}
type is that the \type{Graphic} modifiers and combinators respect the
structure of the graphic (see section~\ref{combining}).  For example,
the \fun{withBrush} modifier only affects the color of the
\type{Graphic} it is applied to, it does not affect the color of the
\type{Graphic} it is embedded in.  In contrast, the \type{Draw} monad
provides operations which change the effect of subsequent drawing
operations.  The following operations correspond to the graphics
modifiers described in section~\ref{modifiers}.

\begin{verbatim}
> selectFont       :: Font          -> Draw Font  
> setTextColor     :: RGB           -> Draw RGB
> setTextAlignment :: Alignment     -> Draw Alignment
> setBkColor       :: RGB           -> Draw RGB
> setBkMode        :: BkMode        -> Draw BkMode
> selectPen        :: Pen           -> Draw Pen  
> selectBrush      :: Brush         -> Draw Brush
\end{verbatim}

These operations all have a type of the form \type{\tva \arrow Draw
\tva}.  The value returned is the old value of the attribute being
changed and can be used to restore the attribute to its previous
value.  For example, the \fun{withFont} modifier could be implemented
like this:

\begin{verbatim}
> withFont new g = do
>   old <- selectFont new
>   g
>   selectFont old
>   return ()
\end{verbatim}

\begin{aside}
This pattern of use is very common in imperative programs so the
Haskell \module{IO} library provides two combinators which encapsulate
this behavior.  The \fun{bracket} function takes three operations as
arguments: a pre-operation \var{left}, a post-operation \var{right}
and an operation \var{middle} and performs them in the order
\var{left}; \var{middle}; \var{right}.  The arguments are provided in
the order \var{left}, \var{right}, \var{middle} because the \var{left}
and \var{right} operations are often ``inverses'' of each other such
as \var{openFile} and \var{closeFile}.  The \fun{bracket\_} function is
similar and is used when the \var{middle} operation does not require
the result of the \var{left} operation.

\begin{verbatim}
> bracket  :: IO a -> (a -> IO b) -> (a -> IO c) -> IO c
> bracket_ :: IO a -> (a -> IO b) -> IO c        -> IO c
> 
> bracket left right middle = do
>   a <- left
>   c <- middle a
>   right a
>   return c
> 
> bracket_ left right middle = bracket left right (const middle)
\end{verbatim}
\end{aside}

% \ToDo{Should these functions be qualified?}

The graphics library provides similar combinators for the \fun{Draw}
monad:
\begin{verbatim}
> bracket  :: Draw a -> (a -> Draw b) -> (a -> Draw c) -> Draw c
> bracket_ :: Draw a -> (a -> Draw b) -> Draw c        -> Draw c
\end{verbatim}

\begin{aside}
In fact, the \fun{bracket} and \fun{bracket\_} functions do slightly
more than the above description suggests.  Those provided in the
\module{IO} library use Haskell's error-catching facilities to ensure that
the \var{right} operation is performed even if the \var{middle} operation
raises an \type{IOError} whilst those in the Graphics library
use Hugs' exception-handling facilities to ensure that the \var{right}
operation is performed even if the \var{middle} operation raises an
exception.
\end{aside}


Using these combinators, it is trivial to implement the modifiers
described in section~\ref{modifiers}.

\begin{verbatim}
> withFont          x = bracket_ (selectFont       x) selectFont
> withTextColor     x = bracket_ (setTextColor     x) setTextColor
> withTextAlignment x = bracket_ (setTextAlignment x) setTextAlignment
> withBkColor       x = bracket_ (setBkColor       x) setBkColor
> withBkMode        x = bracket_ (setBkMode        x) setBkMode
> withPen           x = bracket_ (selectPen        x) selectPen
> withBrush         x = bracket_ (selectBrush      x) selectBrush
\end{verbatim}


% \ToDo{Can we expose the mkFoo functions in the same way?  Not at the moment!}



%%%%%%%%%%%%%%%%%%%%%%%%%%%%%%%%%%%%%%%%%%%%%%%%%%%%%%%%%%%%%%%%

\bibliographystyle{abbrv}
\bibliography{graphics}
\addcontentsline{toc}{chapter}{References}


%%%%%%%%%%%%%%%%%%%%%%%%%%%%%%%%%%%%%%%%%%%%%%%%%%%%%%%%%%%%%%%%
\appendix
%%%%%%%%%%%%%%%%%%%%%%%%%%%%%%%%%%%%%%%%%%%%%%%%%%%%%%%%%%%%%%%%

\section{Quick Reference}

The exported (stable) interface of the library consists of all symbols
exported from \module{HGL.Graphics.Core} and \module{HGL.Graphics.Utils}.
\module{HGL.Graphics} reexports all symbols exported by these modules
and it is expected that most users will only import \module{HGL.Graphics};
the \module{HGL.Graphics.Core} interface is
aimed solely at those wishing to use the graphics library as a base on
which to build their own library or who find the
\module{HGL.Graphics.Utils} interface inappropriate for their needs.

% \begin{description}

\iffalse
% Use this command to generate the documentation:
% ./gendoc GraphicsColor.hs GraphicsEvent.hs GraphicsFont.hs GraphicsPicture.hs GraphicsRegion.hs GraphicsUtils.hs > appendix.tex
%
\modName{GraphicsColor}

\begin{verbatim}
> data Color 
>   = Black
>   | Blue
>   | Green 
>   | Cyan
>   | Red 
>   | Magenta
>   | Yellow
>   | White
>  deriving (Eq, Ord, Bounded, Enum, Ix, Show, Read)
> colorList  :: [(Color, RGB)]
> colorTable :: Array Color RGB
> withColor  :: Color -> Graphic -> Graphic -- SOE, p51
\end{verbatim}

\modName{GraphicsEvent}

\begin{verbatim}
> data Event 
>   = Key       { char :: Char, isDown :: Bool }
>   | Button    { pt :: Point, isLeft, isDown :: Bool }
>   | MouseMove { pt :: Point }
>   | Resize
>   | Closed
>  deriving Show 
\end{verbatim}

\modName{GraphicsFont}

\begin{verbatim}
> createFont :: Point -> Angle -> Bool -> Bool -> String -> IO Font
> deleteFont :: Font -> IO ()
\end{verbatim}

\modName{GraphicsUtils}

\begin{verbatim}
> openWindow        :: Title -> Size -> IO Window
> clearWindow       :: Window -> IO ()
> drawInWindow      :: Window -> Graphic -> IO ()
> getWindowSize     :: Window -> IO Size
> getLBP            :: Window -> IO Point
> getRBP            :: Window -> IO Point
> getButton         :: Window -> Bool -> Bool -> IO Point
> getKey            :: Window -> IO Char
> getKeyEx          :: Window -> Bool -> IO Char
> withFont          :: Font      -> Graphic -> Graphic
> withTextColor     :: RGB       -> Graphic -> Graphic
> withTextAlignment :: Alignment -> Graphic -> Graphic
> withBkColor       :: RGB       -> Graphic -> Graphic
> withBkMode        :: BkMode    -> Graphic -> Graphic
> withPen           :: Pen       -> Graphic -> Graphic
> withRGB           :: RGB       -> Graphic -> Graphic
> withBrush         :: Brush     -> Graphic -> Graphic
> mkBrush           :: RGB                 -> (Brush -> Graphic) -> Graphic
> mkPen             :: Style -> Int -> RGB -> (Pen   -> Graphic) -> Graphic
> emptyGraphic      :: Graphic
> overGraphic       :: Graphic -> Graphic -> Graphic
> overGraphics      :: [Graphic] -> Graphic
> par               :: IO a -> IO b -> IO (a,b)
> par_              :: IO a -> IO b -> IO ()
> parMany           :: [IO ()] -> IO ()
\end{verbatim}


\else
\modName{HGL.Graphics.Core} % accurate 16/1/2000

\begin{verbatim}
> type Title = String
> type Point = (Int,Int)
> type Size  = (Int,Int)
> type Angle = Double
> type Time  = Word32 -- milliseconds
> data RGB = RGB Word8 Word8 Word8
> data BkMode = Opaque | Transparent
> 
> type Alignment = (HAlign, VAlign)
> -- names have a tick to distinguish them from Prelude names (blech!)
> data HAlign = Left' | Center   | Right'
>  deriving (Enum, Eq, Ord, Ix, Show)
> data VAlign = Top   | Baseline | Bottom
>  deriving (Enum, Eq, Ord, Ix, Show)
> 
> data Style
>   = Solid 
>   | Dash        -- "-------"
>   | Dot         -- "......."  
>   | DashDot     -- "_._._._"  
>   | DashDotDot  -- "_.._.._"  
>   | Null
>   | InsideFrame
> 
> runGraphics      :: IO () -> IO ()
> getTime          :: IO Time
> 
> data Window  
> openWindowEx     :: Title -> Maybe Point -> Size -> 
>                     RedrawMode -> Maybe T.Time -> IO Window
>                  
> closeWindow      :: Window -> IO ()
> getWindowRect    :: Window -> IO (Point,Point)
> getWindowEvent   :: Window -> IO Event
> getWindowTick    :: Window -> IO ()
> maybeGetWindowEvent :: Window -> IO (Maybe Event)
> 
> type Graphic = Draw ()
> setGraphic       :: Window -> Graphic -> IO ()
> getGraphic       :: Window -> IO Graphic
> modGraphic       :: Window -> (Graphic -> Graphic) -> IO ()
> directDraw       :: Window -> Graphic -> IO ()
>                  
> selectFont       :: Font          -> Draw Font  
> setTextColor     :: RGB           -> Draw RGB
> setTextAlignment :: Alignment     -> Draw Alignment
> setBkColor       :: RGB           -> Draw RGB
> setBkMode        :: BkMode        -> Draw BkMode
> selectPen        :: Pen           -> Draw Pen  
> selectBrush      :: Brush         -> Draw Brush
> 
> bracket          :: Draw a -> (a -> Draw b) -> (a -> Draw c) -> Draw c
> bracket_         :: Draw a -> (a -> Draw b) -> Draw c -> Draw c
>                  
> data Font        
> createFont       :: Point -> Angle -> Bool -> Bool -> String -> IO Font
> deleteFont       :: Font -> IO ()
> 
> data Brush
> mkBrush          :: RGB                 -> (Brush -> Draw a) -> Draw a
>                  
> data Pen         
> mkPen            :: Style -> Int -> RGB -> (Pen   -> Draw a) -> Draw a
> createPen        :: Style -> Int -> RGB -> IO Pen
> 
> arc              :: Point -> Point -> Angle -> Angle -> Graphic  -- unfilled
> line             :: Point -> Point                   -> Graphic  -- unfilled
> polyline         :: [Point]                          -> Graphic  -- unfilled
> ellipse          :: Point -> Point                   -> Graphic  -- filled
> shearEllipse     :: Point -> Point -> Point          -> Graphic  -- filled
> polygon          :: [Point]                          -> Graphic  -- filled
> text             :: Point -> String                  -> Graphic  -- filled
> 
> data Region
> emptyRegion      :: Region
> rectangleRegion  :: Point -> Point -> Region
> ellipseRegion    :: Point -> Point -> Region
> polygonRegion    :: [Point] -> Region
> intersectRegion  :: Region -> Region -> Region
> unionRegion      :: Region -> Region -> Region
> subtractRegion   :: Region -> Region -> Region
> xorRegion        :: Region -> Region -> Region
> regionToGraphic  :: Region -> Graphic
> 
> data Event 
>   = Key       { char :: Char, isDown :: Bool }
>   | Button    { pt :: Point, isLeft, isDown :: Bool }
>   | MouseMove { pt :: Point }
>   | Resize
>   | Closed
>  deriving Show 
\end{verbatim}


\modName{HGL.Graphics.Utils} % accurate 16/1/2000

Note that this document repeats the definitions of all the functions
defined in \module{HGL.Graphics.Utils}.


\begin{verbatim}
> -- Reexports HGL.Graphics.Core
>
> openWindow        :: Title -> Size -> IO Window
> clearWindow       :: Window -> IO ()
> drawInWindow      :: Window -> Graphic -> IO ()
> 
> getWindowSize     :: Window -> IO Size
> getLBP            :: Window -> IO Point
> getRBP            :: Window -> IO Point
> getButton         :: Window -> Bool -> Bool -> IO Point
> getKey            :: Window -> IO Char
> getKeyEx          :: Window -> Bool -> IO Char
> 
> emptyGraphic      :: Graphic
> overGraphic       :: Graphic -> Graphic -> Graphic
> overGraphics      :: [Graphic] -> Graphic
> 
> withFont          :: Font      -> Graphic -> Graphic
> withTextColor     :: RGB       -> Graphic -> Graphic
> withTextAlignment :: Alignment -> Graphic -> Graphic
> withBkColor       :: RGB       -> Graphic -> Graphic
> withBkMode        :: BkMode    -> Graphic -> Graphic
> withPen           :: Pen       -> Graphic -> Graphic
> withBrush         :: Brush     -> Graphic -> Graphic
> withRGB           :: RGB       -> Graphic -> Graphic
> 
> data Color 
>   = Black
>   | Blue
>   | Green 
>   | Cyan
>   | Red 
>   | Magenta
>   | Yellow
>   | White
>  deriving (Eq, Ord, Bounded, Enum, Ix, Show, Read)
> 
> colorList         :: [(Color, RGB)]
> colorTable        :: Array Color RGB
> withColor         :: Color -> Graphic -> Graphic
> 
> par               :: IO a -> IO b -> IO (a,b)
> par_              :: IO a -> IO b -> IO ()
> parMany           :: [IO ()] -> IO ()
\end{verbatim}

\subsection{Portability notes}

\begin{itemize}
\item \NotInX{\fun{polyBezier}}
\item \fun{shearEllipse} is implemented by polygons on both Win32 and X11.
\item X11 does not directly support font rotation so \fun{mkFont}
always ignores the rotation angle argument in the X11 implementation
of this library.

\item Many of the font families typically available on Win32 are not
available on X11 (and {\it vice-versa\/}).  In our experience, the
font families ``courier,'' ``helvetica'' and ``times'' are somewhat
portable.
\item
  On Win32, the pen is also used to draw a line round all
  the filled shapes --- so the pen color also affects how
  polygons, ellipses and regions are drawn.

\item
  One of the Win32 ``gotchas'' is that the choice of \type{Style}
  only applies if the width is 1 or less.  With greater widths,
  the pen style will always be \fun{Solid} no matter what you try to
  select.  This problem does not apply to X11.
\item The Bitmap functions are not currently provided in the X11
implementation of this library.
\item \fun{shearBitmap} is supported on Win'NT but not Win'95.
\item \NotInWin{\fun{emptyRegion}}
It is possible to use an empty rectangle region instead
\item \fun{ellipseRegion} is implemented using polygons in the X11
implementation of the library.
\item
  Programmers should assume that the \type{Event} datatype will be extended in the
  not-too-distant future and that individual events may change slightly.
  As a minimum, you should add a ``match anything'' alternative to
  any function which pattern matches against \type{Event}s.

\item
X11 systems typically have three button mice.  Button 1 is used as the
left button, button 3 as the right button and button 2 (the middle
button) is ignored.

\end{itemize}


\fi

% \end{description}

%*ignore
\end{document}
%*endignore

% Local Variables:
% indent-tabs-mode: nil
% End:
